\section{Introduction}
Content Addressable Memory (CAM) is an alternate architecture for memory. Unlike Random Access Memory (RAM) which works by 
performing operations for a word by using a memory address, a CAM is able to select multiple memory addresses based on a word. 
Furthermore, a CAM can perform operations like multi-writing into and reading from multiple selected cells in constant time.
This is the reason CAMs are so useful in routers and bridges as they are a much faster way to search for IP addresses to send packets to.
\\\\  
CAMs are usually available in the form of Application Specific Integrated Chipsets (ASICs) and are used in routers, databases and more.
They can be divided into two broad categories, one being Binary CAMs and the other Ternary. 
Ternary CAMs come with the added flexibility of searching for words with masked bits, which is very similar to the regex for '.'. 
Of course, this comes with added circuity to each cell. 
\\\\ 
The goal of this project was to design a module for an expandable Ternary CAM that is synthesizable on a FPGA using modern HDL tools. 
We also integrated a finite state machine that encapsulated the T-CAM and a Serial UART module. 
This enabled the researchers to directly communicate with the T-CAM from a CPU using serial communication with commands for reading, writing, selecting first '. 
