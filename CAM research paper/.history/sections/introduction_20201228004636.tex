\section{Introduction}
CAM or Content Addressable Memory is an alternate architecture for memory. Unlike RAM (Random Access Memory) which works by 
performing operations for a word by using a memory address, a CAM is able to select multiple memory addresses based on a word. 
Furthermore, a CAM can perform operations like multi-writing into and reading from multiple selected cells in constant time.
This is the reason CAMs are so useful in routers and bridges as they are a much faster way to search for IP addresses to send packets to.
\\\\  
CAMs are usually available in the form of application specific integrated chipsets  
The goal of this project was to design a module for a CAM that is synthesizable on a FPGA using modern HDL tools. 
We also integrated a finite state machine that encapsulated the CAM and a Serial UART module. 
This enabled the researchers to directly communicate with the CAM from a CPU using serial communication. 
