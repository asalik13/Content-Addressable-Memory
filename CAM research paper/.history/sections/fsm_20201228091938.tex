\section{The Finite State Machine}
The FSM is the crux of interacting with the CAM and making it carry out complex algorithms. 
It has numerous states, some of them are sending or receiving data, loading data into the CAM, selecting first, searching, reading, setting the comparand and mask as well as writing. 
The base UART module used is from David Things' Github repository. \cite{uart} This works on a 48MHz clock cycle.
To reduce complexities and avoid additional LUTs usage, we used the same clock speed throughout the project. 
\\\\
As shown in figure 4 (TODO), our FSM acts as a way to carry out procedural tasks while avoiding the limit of 21 ns by linking states together. 
Therefore, a task may trigger several states before it returns to the default state. 
For example, the algorithm for searching has several steps, it comprises of
\begin{itemize}
    \item Setting the comparand 
    \item Setting the mask 
    \item Sending the SET signal 
    \item Sending the SEARCH signal 
\end{itemize}
Notice that the first two steps use several clock cycles as only one byte of data flows each clock cycle. 
This is due to the pipeline design of the UART. 
\\\\
An example of a synchronized algorithm like multi-write is given below:
\begin{itemize}
    \item 
\end{itemize}