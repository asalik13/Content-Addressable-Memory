\section{Introduction}
Content Addressable Parallel Processors (CAPPs) constitute an alternative to the standard von Neumann architecture, featuring parallel-processing based on content-addressable memory. Unlike Random Access Memory (RAM) which works by performing operations on a word in memory by referring to its physical address, a CAPP is able to select multiple memory locations simultaneously based on pattern matching the contents of the memory being accessed. Consequently, a CAPP can perform operations like writing into and reading from multiple selected memory cells in constant time. The original intent of the CAPP design was to serve as a general purpose computer, capable of parallel processing. Furthermore, it was hoped that, by providing native support for parallel pattern-matching and multi-write capability, the software for such machines would be relatively easy to write (at least in comparison with other parallel architectures). In practice, this did not occur and CAPPs eventually found use primarily in a much more limited form of application specific CAMs, primarily in the area of computer networking devices, e.g. in the form of fast lookup tables used in network switches and routers.

The goal of our project was to expose a true CAPP capable of multi-write and parallel read, as a convenient USB peripheral, to enable renewed experimentation with this class of architecture. We therefore designed a Verilog module for a parameterized (hence, scalable) CAPP. To this module we added a USB/UART interface module which we manage using an FSM-based protocol. The combined system was implemented on a TinyFPGA-BX \cite{tinyfpga_bx} which we then control over the USB/UART using a simple Python-based driver. We believe that CAPPs have potential in graph theory, neural network caching, high-speed indexing, regex computations etc. and we hope that an open-source, expandable CAPP design which can be synthesized on an FPGA using open-source tools, can provide a basis for renewed research into the application of CAPPs to these diverse and challenging domains. Our implementation is fully open source and can be found here \cite{CAPP_FPGA}.
