\section{Discussion}
Historically, CAPPs have clearly lost out to von Neumann machines. Arguably, this was because the increased price associated with the more complex memory cells was too prohibitive, especially in the early days of computing. However, over the years, manufacturing and designing costs have plummeted while the difficulty of writing correct parallel software has remains as high as it ever was. This makes it a good time to reconsider alternative architectures, which, by exposing powerful primitives such as content addressable read and write operations should simplify the development of parallel algorithms. We have therefore revived Caxton Foster's CAPP research project in the form of an FPGA-based co-processor, which we think will be able to support new, interesting and competitive implementation of algorithms in areas such as: fast lookup tables, neural networks, parallel regex operations, graph algorithms and much more. 